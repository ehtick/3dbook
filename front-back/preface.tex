%!TEX root = ../3dbook.tex

% About this book
% from a course, bundle of lecture notes
% link to videos
% acknowledgements


\chapter*{Preface}

This book is the bundle of lecture notes written for the course \emph{3D modelling of the built environment} (GEO1004), which is part of the MSc in Geomatics at the Delft University of Technology.
Each chapter corresponds to a lesson of the course, and each lesson is accompanied by a short video introducing the key ideas and/or explaining some parts of the lessons.
This book, the videos and other materials are freely available online on the website of the course:

\url{https://3d.bk.tudelft.nl/courses/geo1004/}

\paragraph*{Contents}
The book describes the main ways in which the built environment is modelled in three dimensions, covering material from low-level data structures for generic 3D data to high-level semantic data models for cities.

\paragraph*{Who is this book for?}
The book is written for students in Geomatics at the MSc level, but we believe it can be also used at the BSc level.
The main prerequisites are GIS and programming.

\paragraph*{Acknowledgements.}
We would like to thank Francesca Noardo, who contributed significant parts of the BIM chapter.
Also thank you to all the students who helped us by pointing out errors and typos, especially Zhaiyu Chen and Bingshiuan Tsai.




